\documentclass{article}

\title{Build Notes for LARC and MyPyLARC}

\begin{document}
\maketitle

\begin{abstract}
This document contains build notes for
the LARC software package and the
example package MyPyLARC.
\end{abstract}

\section{Executive Summary: Ubuntu}
On Ubuntu, the following commands can be used to install the
software pre-requisites for LARC and MyPyLARC.
They have been used on Ubuntu 18.04, but should very likely work
for many versions of Ubuntu.
\begin{verbatim}
    # for make, gcc commands
    sudo apt install build-essential

    # for git access to the LARC repository
    sudo apt install git

    # for GNU Multiple Precision arithmetic library
    sudo apt install libgmp-dev

    # for Multiple Precision Floating Point library
    sudo apt install libmpfr-dev

    # for Multiple Precision Complex library
    sudo apt install libmpc-dev

    # for Screen Handling and Terminal Information libraries
    sudo apt install libncurses5-dev

    # for "Python.h" file used by swig
    sudo apt install python3-dev

    # for Python interface/wrapper to C code
    sudo apt install swig

    # for Doxygen source code documentation
    sudo apt install doxygen

    # packages for Python example code
    sudo apt install python3-numpy
    sudo apt install python3-scipy
    sudo apt install python3-matplotlib

    # for Jupyter example code
    sudo apt install jupyter
    sudo apt install python3-pyside
\end{verbatim}

\noindent
Once the software pre-requisites have been installed, the
following commands can be used to obtain the LARC
and MyPyLARC source code from GitHub:
\begin{verbatim}
    git clone https://github.com/LARCmath/LARC.git

    git clone https://github.com/LARCmath/MyPyLARC.git
\end{verbatim}

\noindent
Once the software pre-requisites have been installed
and the source code has been obtained from GitHub, the
following commands can be used to build LARC:
\begin{verbatim}
    # Go into the high-level LARC directory
    cd LARC

    # Select the appropriate local Makefile configuration file
    cp local/Makefile.conf.ubuntu1804 local/Makefile.conf

    # Perform unit tests
    make unittests

    # Build the version of LARC for your desired scalar type
    make clean
    make TYPE=COMPLEX
\end{verbatim}

\noindent
Once the software pre-requisites have been installed
and the source code has been obtained from GitHub, the
following commands can be used to build MyPyLARC:
\begin{verbatim}
    # Go into the high-level MyPyLARC directory
    cd MyPyLARC

    # Select the appropriate local Makefile configuration file
    cp local/Makefile.conf.ubuntu1804 local/Makefile.conf

    # Obtain the initial local copy of LARC
    # NOTE: This command will fail, but it will still have done
    #   what we need it to do before it fails.
    make

    # Select the appropriate local Makefile configuration file
    #   for the local copy of LARC
    cp larc/local/Makefile.conf.ubuntu1804 larc/local/Makefile.conf

    # Build the version of MyPyLARC for your desired scalar type
    make clean
    make TYPE=COMPLEX
\end{verbatim}


\section{Executive Summary: CentOS 8}
On CentOS 8, the following commands can be used to install the
software pre-requisites for LARC and MyPyLARC.
Please contact the developers if there are problems; they have
not been tested yet.
\begin{verbatim}
    yum install git             

    yum install gmp-devel
    yum install mpfr-devel

    # You need to download: "https://ftp.gnu.org/gnu/mpc/mpc-1.1.0.tar.gz"
    tar zxvf mpc-1.1.0.tar.gz
    cd mpc-1.1.0
    ./configure
    make
    make check
    make install

    yum install ncurses-devel
    yum install python3-devel
    yum install swig
    dnf --enablerepo=PowerTools install doxygen
\end{verbatim}

\noindent
After this step, switch over to the Ubuntu instructions.
Use the Makefile.conf.ubuntu1804 local configuration file as
we do not yet have a CentOS local configuration file.


\section{Details: Software Dependencies}
This section describes the various software packages
that LARC and MyPyLARC depend upon.
Note that all the installation commands given
below should be done as the {\tt root} user.

\subsection{git}
\subsubsection{Use}
We use git to access the LARC repository.
In MyPyLARC, the makefile uses git to pull the
most recent version of LARC.
\subsection{Version Requirements}
All the versions of git we have tried work fine;
there are no known restrictions.
In particular, the following versions of git work:
\begin{itemize}
\item git version 1.7.1, released 2010
\item git version 1.8.3.1, released 2013
\item git version 2.17.1, released 2018
\end{itemize}
\subsubsection{General Information}
As of June 2020, the latest version of git is 2.27.0,
released June 1, 2020.
\subsubsection{Installation}
On Ubuntu, use {\tt "apt install git"}.
On CentOS 8, use {\tt "yum install git"}.

\subsection{make}
\subsubsection{Use}
We use make to build LARC and/or MyPyLARC.
\subsection{Version Requirements}
All the versions of git we have tried work fine;
there are no known restrictions.
In particular, the following versions of GNU Make work:
\begin{itemize}
\item GNU Make 3.81, released 2006
\item GNU Make 3.82, released 2010
\item GNU Make 4.1, released 2014
\end{itemize}
\subsubsection{General Information}
As of June 2020, the latest version of GNU Make is 4.3,
released January 19, 2020.
\subsubsection{Installation}
On Ubuntu, use {\tt "apt install build-essential"}.

\subsection{gcc}
This is the GNU Compiler Collection.
\subsubsection{Use}
We use gcc to compile the LARC and MyPyLARC C code,
including the C code generated by SWIG.
\subsection{Version Requirements}
In the gcc 6 series, the following versions of gcc work:
\begin{itemize}
\item gcc version 6.4.0, released 2017
\end{itemize}
In the gcc 7 series, the following versions of gcc work:
\begin{itemize}
\item gcc version 7.5.0, released 2019
\end{itemize}
In the gcc 8 series, the following versions of gcc work:
\begin{itemize}
\item gcc version 8.3.0, released 2019
\end{itemize}
In the gcc 9 series, the following versions of gcc work:
\begin{itemize}
\item gcc version 9.3.0, released 2020
\end{itemize}
Version 10.1 of GCC does not work.  The code compiles,
but it does not build.
\subsubsection{General Information}
As of June 2020, the latest version of GCC is 10.1,
released on May 9, 2020.
\subsubsection{Installation}
On Ubuntu, use {\tt "apt install build-essential"}.
\subsubsection{Makefile Considerations}
The compiler must be on the executable path and accessable
via the command \texttt{"gcc"}.

\subsection{gmp}
This is the GNU Multiple Precision Arithmetic Library.
\subsubsection{Use}
This library is used to implement LARC's multi-precision
scalar types.
Note that even if LARC is compiled with a fixed-precision
scalar type (i.e., INTEGER, REAL, COMPLEX), this library
is still needed internally to compute pre-defined scalar
values.
\subsection{Version Requirements}
We need version 5.0 or later of GNU GMP because it
is required by MPFR version 4.0.0.
In particular, the following versions of GNU GMP work:
\begin{itemize}
\item gmp version 6.1.2, released 2016
\item gmp version 6.2.0, released 2020
\end{itemize}
\subsubsection{General Information}
As of June 2020, the latest version of GMP is 6.2.0,
released January 17, 2020.
\subsubsection{Installation}
On Ubuntu, use {\tt "apt install libgmp-dev"}.
On CentOS 8, use {\tt "yum install gmp-devel"}.
\subsubsection{Makefile Considerations}
The location of the include file \texttt{"gmp.h"}
is specified by the variable \texttt{GMPIDIR} in the
local Makefile configuration file \texttt{"local/Makefile.conf"}.
The location of the library file \texttt{"libgmp.so"}
is specified by the variable \texttt{GMPLDIR} in the
local Makefile configuration file \texttt{"local/Makefile.conf"}.

\subsection{mpfr}
This is the GNU Multiple Precision Floating-Point Reliable Library.
\subsubsection{Use}
This library is used to implement LARC's multi-precision
scalar types.
Note that even if LARC is compiled with a fixed-precision
scalar type (i.e., INTEGER, REAL, COMPLEX), this library
is still needed internally to compute pre-defined scalar
values.
\subsection{Version Requirements}
We need version 4.0.0 or later of GNU MPFR due to our use of
the function {\tt "mpfr\_get\_q()"}.
In particular, the following versions of GNU MPFR work:
\begin{itemize}
\item mpfr version 4.0.2
\end{itemize}
\subsubsection{General Information}
As of June 2020, the latest version of MPFR is 4.0.2,
released January 31, 2019.
\subsubsection{Installation}
On Ubuntu, use {\tt "apt install libmpfr-dev"}.
On CentOS 8, use {\tt "yum install mpfr-devel"}.
\subsubsection{Makefile Considerations}
The location of the include file \texttt{"mpfr.h"}
is specified by the variable \texttt{MPIDIR} in the
local Makefile configuration file \texttt{"local/Makefile.conf"}.
The location of the library file \texttt{"libmpfr.so"}
is specified by the variable \texttt{MPLDIR} in the
local Makefile configuration file \texttt{"local/Makefile.conf"}.

\subsection{mpc}
This is the GNU Multiple Precision Complex Library.
\subsubsection{Use}
This library is used to implement LARC's multi-precision
scalar types.
Note that even if LARC is compiled with a fixed-precision
scalar type (i.e., INTEGER, REAL, COMPLEX), this library
is still needed internally to compute pre-defined scalar
values.
\subsection{Version Requirements}
We need version 1.1.0 of GNU MPC due to our use of
the function {\tt "mpc\_rootofunity()"}.
In particular, the following version of GNU MPC works:
\begin{itemize}
\item mpc version 1.1.0
\end{itemize}
\subsubsection{General Information}
As of June 2020, the latest version of GNU MPC is 1.1.0,
released January 2018.
\subsubsection{Installation}
On Ubuntu, use {\tt "apt install libmpc-dev"}.
On CentOS 8, the command is {\tt "dnf --enablerepo=PowerTools install libmpc-devel"},
except that it doesn't install a recent enough version.

\medskip
\noindent
To build and install the latest version:
\begin{verbatim}
    # download "https://ftp.gnu.org/gnu/mpc/mpc-1.1.0.tar.gz"
    tar zxvf mpc-1.1.0.tar.gz
    cd mpc-1.1.0
    ./configure
    make
    make check
    make install
\end{verbatim}
\subsubsection{Makefile Considerations}
The location of the include file \texttt{"mpc.h"}
is specified by the variable \texttt{MPIDIR} in the
local Makefile configuration file \texttt{"local/Makefile.conf"}.
The location of the library file \texttt{"libmpc.so"}
is specified by the variable \texttt{MPLDIR} in the
local Makefile configuration file \texttt{"local/Makefile.conf"}.

\subsection{curses}
\subsubsection{Use}
\subsection{Version Requirements}
\subsubsection{General Information}
\subsubsection{Installation}
On CentOS 8, use {\tt "yum install ncurses-devel"}.
\subsubsection{Makefile Considerations}
...

\subsection{python3}
\subsubsection{Use}
\subsection{Version Requirements}
\subsubsection{General Information}
\subsubsection{Installation}
On CentOS 8, use {\tt "yum install python3-devel"}.
\subsubsection{Makefile Considerations}
The main command must be on the executable path and accessable
via the command \texttt{"python3"}.
The include file \texttt{"Python.h"} must be in one of
include directories specified in the main Makefile or local Makefile
configuration file.

\subsection{swig}
\subsubsection{Use}
\subsection{Version Requirements}
We need SWIG 3.0.0 or later for Python3 support.
In particular, the following versions of SWIG work:
\begin{itemize}
\item SWIG version 3.0.12
\item SWIG version 4.0.1
\end{itemize}
\subsubsection{General Information}
As of June 2020, the latest version of SWIG is 4.0.2,
released June 8, 2020.
\subsubsection{Installation}
On Ubuntu, use {\tt "apt install swig"}.
On CentOS 8, use {\tt "yum install swig"}.
\subsubsection{Makefile Considerations}
The main command be on the executable path and accessable
via the command \texttt{"swig"}.

\subsection{doxygen}
\subsubsection{Use}
\subsection{Version Requirements}
We know that doxygen version 1.8.5 basically works,
except that it generates many ``unsupported tag''
warnings about the {\tt "src/Doxyfile.in"} file.
The following warning is harmless and may be ignored:
\begin{verbatim}
Warning: ignoring unsupported tag ..., file src/Doxygen.in
\end{verbatim}
for the following tags:
\begin{verbatim}
... `ALLOW_UNICODE_NAMES    =' at line 79, ...
... `TOC_INCLUDE_HEADINGS   =' at line 313, ...
... `GROUP_NESTED_COMPOUNDS =' at line 370, ...
... `HIDE_COMPOUND_REFERENCE=' at line 541, ...
... `SHOW_GROUPED_MEMB_INC  =' at line 554, ...
... `WARN_AS_ERROR          =' at line 765, ...
... `LATEX_EXTRA_STYLESHEET =' at line 1743, ...
... `LATEX_TIMESTAMP        =' at line 1810, ...
... `RTF_SOURCE_CODE        =' at line 1876, ...
... `MAN_SUBDIR             =' at line 1911, ...
... `DOCBOOK_PROGRAMLISTING =' at line 1974, ...
... `DIA_PATH               =' at line 2168, ...
... `DIAFILE_DIRS           =' at line 2380, ...
... `PLANTUML_JAR_PATH      =' at line 2388, ...
... `PLANTUML_CFG_FILE      =' at line 2393, ...
... `PLANTUML_INCLUDE_PATH  =' at line 2398, ...
\end{verbatim}

Additionally, the following versions of doxygen work without warnings:
\begin{itemize}
\item doxygen version 1.8.13
\item doxygen version 1.8.17
\end{itemize}
\subsubsection{General Information}
As of June 2020, the latest version of Doxygen is 1.8.18,
released April 13, 2018.
\subsubsection{Installation}
On Ubuntu, use {\tt "apt install doxygen"}.
On CentOS 8, use {\tt "dnf --enablerepo=PowerTools install doxygen"}.
\subsubsection{Makefile Considerations}
The main command must be on the executable path and accessable
via the command \texttt{"doxygen"}.

\end{document}
